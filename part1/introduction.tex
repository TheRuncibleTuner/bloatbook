
\chapter{Introduction}

Managing your Java program's memory couldn't be easier, or so it would
seem. Java provides you with automatic garbage collection, and a compiler that
runs as your program is running and responds to its operation. There are many
standard, open source, and proprietary libraries and frameworks available,
all written by experts. All you have to do is piece together the parts and let
the Java runtime system do the rest for you.

The reality, unfortunately, is very different. If you just assemble the parts, take
the defaults, and follow all the good advice to make your
program flexible and maintainable, you will likely find that your memory needs are
\emph{much} higher than imagined. You may also find that precious memory resources
are wasted on data that is sticking around longer than necessary, or that 
CPU resources are wasted recreating data that could have been saved longer. 

This book is a guide to using memory well in Java.  While there has been much written on 
how to build systems that are easily maintained and extended, bug-free, and secure, there is
little guidance available on efficient use of memory.  Unfortunately memory usage
is often an afterthought, where much is left to chance, at least
until there is a crisis. The theme of this book is that memory usage,
like any other aspect of software, needs to be carefully engineered if your system is to be successful. 
This book provides an overall approach to engineering memory usage, and a practical guide to 
making informed decisions at every step of the way. (every part of a design)

Engineering memory usage properly, and early in the cycle, can make the difference between success and failure a 
software project. If at system test or deployment you find that your design does not fit in memory, ...
There are many reasons why in Java systems, good memory usage requires can require more
attention and be more difficult to achieve, than in some other languages.  Unfortunately,
the consequences of not getting the design right can be worse as well.
The next section explores (motivates?) these technical realities
that make good use of memory both essential and challenging. It also looks at some popular misconceptions that can
get in the way of solid engineering.




not so easy, and is extremely important for your system's success, because of Java and the use of so
many frameworks (current style of programming)

So much is hidden
Expensive compared to other languages
Problems can pile up
Limited choices compared to other languages.

Languages like C give you a lot of control over storage, and so it is clear what
the consequences of your choices are. In Java much is done for you, both by the runtime and by the many layers of frameworks, so it is easy to lose sight of the space your data designs require. In fact, it can be downright difficult to find out. This is especially true early in the development cycle, when it is easier to make design changes.  Java also gives you fewer storage options than languages like C, and some of these options are surprisingly expensive. < In fact, many of the basic building blocks in Java require more space than C++. > So careful attention to the cost of design decisions is especially important in Java.  This book's main purpose is to help you make informed design tradeoffs based on a solid knowledge of common data design options and their space costs.  < And the sooner the better, since once memory problems are discovered they can require extensive
refactoring >. 

There are some common beliefs and there are some common beliefs that can get in the way 
Before getting into too much technical detail, ... The next section dispels some common misconceptions about memory in Java.  These misconceptions can get in the way of making  The remaining sections in this chapter introduce some terminology and conventions used in the rest of the book.

%move a lot of this to preface as motivation/inspiration for book
Thanks to years of software engineering advances, there is now a large
programmer toolkit for producing well-designed and maintainable programs,
including a plethora of modeling techniques, design patterns, and coding 
advice. This is a success story for software engineering. Productivity has
increased at a remarkable pace to meet the demands of an exploding software-driven world. 

Along the way, programming has changed in some fundamental ways. Rather than building applications from scratch, programmers 
assemble applications from a collection of off-the-shelf, reusable frameworks
and libraries. These reusable frameworks and libraries provide suitable
abstractions, while under the covers they do the heavy lifting, performing a
myriad of low-level tasks like connecting to a database, managing transactions,
providing security, or building user interface widgets.

Layers of abstraction make it easy to construct applications
rapidly, but also make it easy to introduce huge inefficiencies without the
programmer being aware of it. Consuming too much memory is a particularly common
kind of inefficiency. When a program stops with an out-of-memory exception, it's
often hard to tell whether there's a memory leak or the program is just too big. 
In either case, the next more difficult
question is how to fix it. This book is about to how to avoid memory problems in the first place. Before
getting into specifics, we first look at the bigger picture, to motivate
why you should care about how much memory your program is using.


\section{The Big Pileup}

A Java application is like an iceberg, and you, the programmer, typically see
only the very tip. Coding at the tip, you select which libraries
and frameworks to use for various purposes. Abstractions provided by these
libraries and frameworks insulate you from the actual implemention. At the tip
of the iceberg, you don't see (and you don't need to see) what's under water,
and this what reusability is all about. Abstractions improve understandabiliy, quality, and
pruductivity. 

Unfortunately, abstractions also hide memory costs, and
 you are unaware whether a framework method call creates 5 objects 
or hundreds of objects. If a prorgram runs out of memory,
then abstractions suddenly get in the way. To fix a memory problem, assuming
you can't just increase the heap size, you need to get inside what's been
carefully hidden from you, the rest of the iceberg. Typically, once you dive in,
you will see that the frameworks you are using call
other frameworks, which call other frameworks, and so on, eventually calling
core libraries. The iceberg is really a big layering of abstractions, large
enough to overwhelm anyone. 

Often a memory problem doesn't have a single cause, which is another reason why
it can be hard to fix. Like developers at the tip of the iceberg, framework
developers also use abstractions without understanding their memory costs, so
that each abstraction layer introduces inefficiencies. Added together, these  
result in a systemic bloat problem, which needs extensive refactoring to
fix. Limiting systemic bloat requires paying attention to memory
 at every step of the development process and at every level of abstraction.


\section{Some Common Misconceptions About Memory}

In addition to the technical reasons why managing memory can be a challenge, there
are other reasons why memory footprint problems are so common. In particular,
the software culture and popular beliefs lead programmers to ignore memory
costs. Some of these beliefs are really myths --- they might have once
been true, but no longer. Here are several.

\textbf{Misconception.} Memory is cheap, hardware is always improving, so
things are fine. \textbf{Reality.} With gigabytes of memory available, you never
need to worry about running out. However, resources are still finite, and it is surprising how easy it is to
saturate them. Furthermore, while processor speeds have been doubling every two
years following Moore's law, a physical limit has now been reached preventing further processor speedup. 
Instead, the number of processor cores on a chip is expected double every two years. However,
memory bandwidth and cache sizes will not grow proportionally. 
 Larger heaps combined with relatively less bandwidth is a recipe for a big performance hit going forward, 
 and the rapidly growing number of small embedded processors will require more
 efficient use of memory.

\textbf{Misconception.} The JIT optimizer and garbage collector are so good that
they will clean up all efficiencies. things are fine, everything is cheap; there was an
article on developer works, called "Go ahead, make a mess",  and this idea that objects are free, at least temporaries are,
 and everything will be taken care of for you. The JIT is going to clean things up, the garbage collector will elp, 
 all of this great research 
  on garbage collection and jit optimization, and you shouldn't have to worry about that; just code whatever the best design 
  is from a maintainability standpoint, or whatever other standpoint, and the performance will magically be taken care of.

In fact, in the memory space, the JIT is doing, all of the commercial JITs that we know of are doing absolutely nothing 
in terms of storage optimization, so that every single object, with all of its fields, ends up as an object in the heap, 
taking up space.  We'll look into the detail of what that means. And similarly the garbage collector, yes it is cleaning up 
temporary objects. But even for temporaries there are other costs.
General belief is that JRE, JIT, garbage collection take care of things.

The construction of the objects in the first place; garbage collectors are only dealing with shortlived objects,
 and footprint problems are a problem of long-lived objects, which garbage collectors are not addressing at all.
 

\emph{Myth}: If you are using a library or framework, in particular if
it is popular, it is natural to think that it was developers by experts, and
therefore it is highly efficient. Therefore, you do not have to worry about
its memory footprint. In reality, framework developers may not know
the costs of the frameworks that they themselves are using.  Additionally, they
can't predict how their framework will be used, and you cannot assume that it
will be optimized for your particular situation.


Myth 4: Performance or memory -- tradeoff -- would sacrifice good design.
Goal of the book:  teach how do good design, while taking memory costs into account.

Many developers know things are bad, but not how had, and we see this over and over again. 
In fact, I just came back, about a month ago, working with a group of IBM rational developers in Ottowa, 
a very strong group of people, very good engineers, who were quite aware of how memory contrained they were,
 and even they were surprised still when we looked at the actual cost, to see just how costly things were. 
 They were even more than they thought.

Then there are the people who just give up - I know Java is expensive, it's always expensive, 
there is nothing I can do about it, it's just a cost of object oriente programming in Java, and if I try to do anything, 
it's going to break my good design.
And so, hopefully, one of the things we want to achieve is to raise awareness of cost. 
So that it's not a lost cause, there is some hope.  It's possible to make informed tradeoffs, can't fix all problems. 
Identify places where good engineering can help by the developer, and then where people are going to hit a wall.

Finally, this is an issue for performance, not just for memory. 
There's a false dichotomy in a lot of people's minds that say well, if you have a lot of footprint it must be buying
 you something in terms of performance. But in reality, that is not always the case. In fact, sometimes bad memory usage
  will result in poor performance as well, even something as simple as I am using so much of my heap for long-lived objects, 
  then I have very little headroom for temporaries.  And so my garbage collector has to run much more often. 

Or I don't have enough room to size the caches for things I get from the database as large as I like, 
so I go back to the database more than I like, so this can have a huge performance cost.

Fortunately, bloated designs are not an inevitable consequence of object-oriented development.

\section{Quiz}

Understanding memory costs requires counting bytes, which may seem like a strange activity for a Java programmer,
 accustomed to rapid assembly of applications from assorted libraries. At its core, programming is an engineering discipline, 
 and there is no escaping the fact that the consumption of any finite resource must be measured and managed.  
 To start you thinking about bytes, here is a quiz to test how good you are at estimating sizes of Java objects. 
 Assume a 32-bit JVM.
 
\begin{verbatim}

   Question 1: What is the size ratio in bytes Integer to int?
   
      a. 1:1
      b. 1.33:1
      c. 2:1
      d. 4:1
      e. 8:1
   
   Question 2: How many bytes in an 8-character string?

      a. 8 bytes
      b. 16 bytes
      c. 20 bytes
      d. 40 bytes
      e. 56 bytes
 
   
   Question 3: Which statement is true about a HashSet compared to 
               a HashMap with the same number of entries?
               
      a. It has fewer data fields and less overhead.
      b. It has the same number of data fields and more overhead.
      c. It has the same number of data fields and the same overhead.
      d. It has more data fields and it has less overhead.
                  
                       
   Question 4: Arrange the following 2-element collections in size order:
    
      ArrayList, HashSet, LinkedList, HashMap
          
   Question 5: How many collections are there in a typical heap?
   
      a. between five and ten
      b. tens
      c. hundreds
      d. thousands
      e. order(s) of magnitude more than any of the above

   Question 6: What is the size of an empty ConcurrentHashMap?
   (Extra Credit)
      a. 17 bytes
      b. 170 bytes
      c. 1700 bytes
      d. 17000 bytes
      e. 500 bytes
           

ANSWERS: 1d, 2e, 3b, 
         4 ArrayList LinkedList HashMap HashSet, 
         5e, 6c                 
\end{verbatim}

If you look inside a typical Java heap, it is mostly filled with the kinds of objects used in the quiz --- boxed scalars, 
strings, and collections. Every time a program instantiates a class, there is an object created in the heap, and as 
shown by the 4:1 size ratio of \texttt{Integer} to \texttt{int}, objects are not cheap. 

Strings often consume half of the heap and are surprisingly costly. 
If you are a C programmer, you might think that an 8-character string should consume 9 bytes of memory, 
8 bytes for characters and 1 byte to indicate the end of the string. What could possibly be taking up 56 bytes? 
Part of the cost is because Java uses th 16-bit Unicode character set, but this accounts for only 16 of the 56 bytes. 
The rest is various kinds of overhead.

After strings, collections are the most common types of objects in the heap. In typical real programs, having 100's of 
thousands and even millions of collection instances in a heap is not at all unusual. If there are a million collections 
in the heap, then the collection choice matters. One collection type might use 20 bytes more than another, 
which may seem insignificant, but in a production execution, the wrong choice can add 20 megabytes to the heap.

\texttt{ConcurrentHashMap}, compared to the more common collections in the standard library, is surprisingly expensive. 
If you are used to creating hundreds of \texttt{HashMaps}, then you might think that it is not a problem to create hundreds 
of \texttt{ConcurrentHashMaps}. There is certainly nothing in the API to warn you that \texttt{HashMaps} and 
\texttt{ConcurrentHashMaps} are completely different when it comes to memory usage. 
The quiz gives some sense of how surprising the sizes are for the Java basic objects, like \texttt{Integers}, 
strings, and collections. When code is layered with multiple abstractions, memory costs become more and more difficult 
to predict.
%Looking at objects individually, what heappens



In fact, as we will see in some of our examples, inside the Java library themselves, the low level libraries are calling
 other low level libraries.  String buffer uses ;;  Hashset interms of hashmap, and so forth.  
 This is true for memory and performance.

Compared to systems languages like C, Java space costs are high, even for the most basic building blocks. 
This makes it all the more important for developers to be aware of memory costs.


\section{Notation and Conventions}

\subsection{Entity-Collection Diagrams}

Much of this book is about how to implement your data designs to make the most efficient use of space. In this section we introduce a diagram, called the \emph{entity-collection(E-C) diagram}, that helps with that process. It highlights the major elements of the data model implementation, so that the costs and scaling consequences of the design are easily visible. We use these diagrams throughout the book to illustrate various implementation options and their costs.

A data model implementation begins with a conceptual understanding of the entities and relationships in the model.  This may be an informal understanding, or it may be formalized in a diagram such as an E-R diagram or a UML class diagram.  At some point that conceptual model is turned into Java classes that represent the entities, attributes, and assocations of the model, as well as any auxiliary structures, such as indexes, needed to access the data.  The example below shows a simple conceptual model, using a UML class diagram.  A Java implementation of that model is also shown, using rectangles for classes and arrows for references.  %The Java diagram below is typical of a schema diagram, in that it shows 

While the Java 


In these models we make a distinction between the implementations of entities and the implementation of collections.  We do this for a number of reasons.  First, the kinds of choices you make to improve the storage of your entities are often different from those Collections are also depicted as nodes, using an octagonal shape.  This is different from UML class diagrams, where associations are shown as edges. E-C diagrams show 

\subsection{Defining Terms}

Terms like object can have different meanings in the literature.  The following are the conventions for terms used throughout this book.

Since the word \textit{object} can have different meanings, we precisely define the terminology used:
\begin{itemize}
\item A \textit{class} is a Java class. A class name, for example \texttt{String}, always appears in type-writer font. 
\item A \textit{data model} is a set of classes that represents one or more logical concepts.
\item Finally, an \textit{object} is an instance of a class, that exists at runtime occupying a contiguous section of memory.
\end{itemize} 







 

