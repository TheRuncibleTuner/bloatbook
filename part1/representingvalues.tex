\chapter{Representing Field Values}
\section{Character Strings}
\paragraph{Strings vs. Compiled Forms}
\paragraph{StringBuffer vs. String}

It is well-known that \class{StringBuffer} is more efficient than
\class{String} for performing string concatenation. Since \class{String}s are
immutable, concatenating \class{String}s involves allocating a temporary
\code{char} array, copying the \class{String}s into it, and then constructing
a result \class{String}. A \class{StringBuffer}, on the other hand, is mutable.
If the \class{StringBuffer} capacity is sufficient, then \class{String}s
can be concatenated by simply appending them to the \class{StringBuffer}.

However, long-lived \class{StringBuffer}s can waste memory. Usually a
\class{StringBuffer} is 40\% empty space, since they double in size whenever
they need to be reallocated. Typically, after a string is built up in the
\class{StringBuffer}, it is stable, at which point it should be converted 
to a \class{String}, so that the \class{StringBuffer} can be
garbage collected. Using \class{StringBuffer}s to facilitate
building a \class{String} is fine, but they should be used only as temporaries.

\section{Representing Bit Flags}
\label{sec:bit-flags}

\section{Dates}
\section{BigInteger and BigDecimal}