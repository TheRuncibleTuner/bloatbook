\chapter{Collections: An Introduction}
\label{chapter:brief-introduction-collections}

Collections are the glue that bind your data together.  Whether 
providing random or sequential access, or enabling
look up by value, collections are an essential part of any
design. In Java, collections are very easy to use, and, just as easily, to misuse when
it comes to space. Like much else in Java, they don't come with a price tag
showing how much memory they need. In fact, collections
often use much more memory than you might expect. In most Java applications they are the second largest consumer of memory, after
Strings. It's not unusual for collection overhead to take up between x and y\%
of the Java heap. The way collections are employed can make
or break a system's ability to scale.

%awareness of costs; thorough analysis of scalability; whether or not will scale

This chapter and the next four chapters are about using
collections in a space-efficient way.  Collections may serve a number of very
different purposes in an application. They can be used to implement
relationships, to organize data into tables, to enable quick lookup via indexes, or
to hold informal annotations on more formally modeled data. Each use
has its own best practices, as well as traps to avoid. The next four chapters are an in-depth
look at each of the different ways collections can be used. Each chapter
takes you through typical patterns of usage, and helps you understand what
collections cost, and gives you techniques for analyzing how
small decisions play out at a larger scale. Understanding your requirements
to match a collection to the task, to avoiding paying for functionality you
don't need.  Here is a roadmap of the next four chapters:
% common traps and
% best practice solutions to help you avoid memory problems. 

%Shows you costs
%Importance of scalability analysis
%Understanding requirements can lead to specialized solutions
%Look inside a few important collection classes
%


\begin{description}
\item[Chapter~\ref{chapter:representing-relationships}. Representing
Relationships] This chapter covers using 
collections to implement relationships that enable quick
navigation from an object to related objects via references. At runtime,
each relationship turns into a large number of collection instances, many containing just a few elements each.
%In a product catalog, for example, each product object would have one
%collection instance pointing to that product's suppliers, and another pointing
%to the product's parts.
The main issues to watch for in these designs
are keeping the cost of small and empty collections to a minimum, sizing
collections properly, and paying only for functionality that is truly needed.

\item[Chapter~\ref{chapter:tables-indexes}. Tables and Indexes] This
chapter covers collections that serve as the jumping off point for accessing
objects. This could be a table that provides direct access to all the
objects of one type, or an index for looking up objects by unique key. These are
usually large structures. In more complex cases such as multikey and multivalue maps,
there is a choice between a single collection vs. a multilevel structure.
This chapter covers predicting memory costs as the structure
grows.
%understanding which costs are amortized and which are not 

\item[Chapter~\ref{chapter:dynamic-records}. Attribute Maps and Dynamic
Records] Many applications need to represent data whose shape is not
known at compile time. This data might be property-value pairs
read from a configuration file at runtime, or database records returned from
a dynamic query. Since Java does not let you define new classes on the
fly, collections are a natural, though costly way to represent these
dynamic records. This chapter looks at the common cases where dynamic records
are needed, and shows how to identify properties of your data that could lead to
more space-efficient solutions.

\item[Chapter~\ref{chapter:additional-collection-behaviors}. Additional
Collection Behaviors] Collections sometimes need to supply functionality
beyond their main purpose of storing and accessing data. For example,
you may want to protect a table from inadvertent
updates, or minimize contention on a map when running in a
multithreaded environment. Java collections provide these capabilities through
specialized classes or through wrappers that alter
the function of existing collection classes. This chapter covers the costs of
this functionality. A key issue is that the depth
at which these features are used can have a big impact on the memory
cost.

\end{description}

Across all of these different cases, collections have some common themes.
The rest of this chapter is an overview of issues and resources to be aware of
in any use of collections. The next section is a quick
introduction to the memory costs of collections. It is followed by a brief
survey of collections resources available in the
standard libraries and in some open source alternatives.

\section{The Cost of Collections}
%\section{Designing with Collections}
\label{sec:designing-with-collections}

%costs are high
For example,
the very smallest standard collection is 40 bytes, an \emph{empty}
\class{ArrayList}, and that's only when it's been carefully initialized. 
Otherwise, by default it takes up 80 bytes. That may not sound like a lot,
but if you have a collection deeply nested in your design, there may be
hundreds of thousands or millions of instances.

The cost of different collections can vary greatly, even among choices that
might work equally well in the same situation. For example, a 5-element \class{ArrayList}
with room for growth takes 80 bytes, while an equivalent \class{HashSet} costs 276 bytes, or 3.5 times as much. 
Like other kinds of infrastructure, collections serve a necessary
function, and paying for overhead can be worthwhile.  That is, as long as
the cost is proportional to the benefit, and you are not paying
for functionality you don't need. In the above example, a 70\% space saving
can be achieved if the application can do without the uniqueness checking
provided by \class{HashSet}.  The good news is that there are often a few small
changes you can make to the way you use collections, which can have a big
impact on their footprint.

At the same time, there is much that is not under your control, and the basic
costs of collections are high.  Because the standard collection libraries are
written in Java, they suffer from the same kinds of bloat we've already encountered. They have internal layers
of delegation, and extra fields for features that your program may not
use. Unfortunately, the standard collections provide few levers for
tuning for each individual situation. They were
mostly designed for speed, and space is usually sacrificed. In
addition, the standard APIs can force you into expensive decisions in other
parts of your design, such as requiring you to box scalars that you place in
collections. Given all of this, it is important to understand, as early as possible,
what collections cost in your implementation. In this and
the next four chapters we show you how to figure out those costs, and how
to avoid some common traps. Armed with this
knowledge you can choose carefully, ensuring that you are not
overpaying for your system's needs.
  
It is not enough to know the cost
of collections in the abstract, but to understand \emph{how they will work in
your design}. This is because the same collection class can have very different
scaling behavior in different situations. For example, a particular map
may be perfect as an index over a large table containing a
million objects. That same map class, however, may be prohibitively expensive
when deeply nested, for example, when each of those
million objects maintains its own side map to store attribute-value pairs that
annotate the object.
%make example more concrete? 
Collections have fixed and variable costs, as we saw
in Section~\ref{scalability}. These costs play out differently
depending on the context in which a collection appears, for example, whether
there are many small nested collections, a few large ones, or some combination
hybrid.  A given collection choice has as a minimum its fixed cost,
whether or not it contains any elements. Fixed costs will add up depending on
how many collection instances there are. The variable cost is the space needed to store each entry. These
 costs accumulate with the total number of
elements in collections, regardless of whether they are spread across a lot of
small collections or concentrated in a few large ones. 
Some collection costs will be amortized as more data is stored, others will only
continue to grow. Collection costs also need to be considered together with the data they are
storing. If you are using very expensive collections to store very small amounts
of data, than your system will have a high overhead ratio, and ultimately your application's ability
to scale will be limited. 
Knowing how to analyze these costs in the context of your design
is the key to ensuring that your system will scale when it is time for
deployment.  
%That is the focus of the next four chapters.
%certain collection classes were designed only to operate at a certain scale

\section{The Standard Java Collections}


\section{Alternative Collections Frameworks}
In addition to the standard Java collections classes, there are a number of
open source collections frameworks available. Many of these frameworks
are designed to give you additional functionality or make it easier to
program. With some exceptions, most are not designed to be efficient from a
space standpoint. While some do make much better use of space, and give you a
better ability to customize for your situation, others actually have a
large footprint for the same functionality as the standard libraries. So as in
anything else, being an informed consumer (by measuring) is the best way to
avoid problems later. Below is a short summary of the most
popular alternative collections frameworks.

Important note: there are many kinds of open source licenses. All have some
restrictions on usage. Make sure to check with your organization's open
source software policies to see if you may use a specific framework in 
your product, service or internal system.

(TODO: Subsections go here for each major framework. Short summary of each.
Longer section on any framework that has customization of policies or that lets you override
entries. Also a brief subsection on collection classes within other frameworks.)

\section{Summary}
Using collections carefully can
make the difference between a design that scales well and one that
doesn't. Some items to be aware of when working with collections:
\begin{itemize}
  \item The standard Java collections, along with most alternative
  collections, were not designed with space in mind. Nor are they optimized 
  for some very common use cases, such as designs with many small collections.
  In general the Java collections use a lot of memory.
 %not designed for the ways they are actually used.
  \item Collections vary widely in their memory usage. Awareness of
  costs is an essential first step in choosing well. Sometimes there is a less
  expensive choice of collection class available. Initialization options
  can also make a big difference.
  \item The same collection class will scale differently depending on its
  context. Ensuring scalability requires
  analyzing how a collection's fixed and variable costs
  play out in a given situation. A given collection choice may not be appropriate for its level of nesting.
  \item Analyzing your application's requirements, such as 
  how much of a collection's functionality is really needed, can lead to
  specialized, space-saving solutions.
\end{itemize}
