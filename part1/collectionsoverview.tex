\chapter{A Brief Overview of Collections}
Collections are the glue that connect your data together. In
many Java heaps they are the second largest consumer of memory, after Strings. 
It is pretty common to see applications where collections use x-y\% of
the memory. By the way, we use the term collections
broadly, to cover random and sequential access structures, such as lists and sets, as well as
associative structures such as maps and property-value pairs.

Like much else in Java, collections are very easy to take
``off the shelf'' and use. Meanwhile, their costs are not obvious. 
In fact, collections often use much more memory than one would expect. The very
smallest standard collection is 40 bytes, an
\emph{empty} \class{ArrayList} when carefully configured. The cost of
different collections can vary greatly, even among choices that could work just
as well in the same design. For example,
%Give a numeric example here instead
Remember that while
collections serve an important function, it's an adminstrative one.  
Administrative overhead can be worthwhile, as long as
the cost is reasonable for the benefit attained. 

The good news is that a few small programming
changes or configuration settings can often have a big impact. At the same
time, there is much that is not under your control.  

One
goal of the next few chapters is to help you become aware of these costs, 
so that you are not paying for functionality that you don't need.
The next few
chapters will also help you evaluate your overall design for scalability, early
enough when design changes are still possible.  In chapters x through y we'll go
through how you use collections in your design.  But first, in the rest of this
chapter we take a quick look at what's available and things to watch
out for (reword!).

Collections are used for many purposes in Java applications. In the
next four chapters we walk you through the common ways collections are used,
pointing out traps and some space-saving best practices. First, in this
chapter, we take a quick tour of the issues you are likely to face when
using collections, and give you some basic guidelines. We also give you a guide to
some available collection resources.



\section{The Costs of Collections}

Why are the Java collections expensive? First of all,
they are implemented in Java, and so they suffer from the
same kinds of bloat we've seen in other Java structures. Many have internal layers of delegation, and extra fields
for functionality that your program doesn't need. Also, the standard collections
were designed with speed as a priority, and in many cases space was
sacrificed.
%So just because experts designed the collections, that doesn't mean
%they are efficient users of space




Table shows the minimum fixed size.  In fact, it can be even bigger if the
collection is not configured carefully.
\paragraph{Relationships}  One of the most common uses of collections in Java is
to let you navigate from an object to its related objects, using references.
\paragraph{Tables and Indexes}
\paragraph{Dynamic Records and Property Maps}
\paragraph{Orthogonal Behaviors}

\section{The Standard Java Collections}

\section{Alternative Collections Frameworks}

\section{Summary}

