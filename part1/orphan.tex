% this is extra text, mostly duplicate writeups

% The following sections are already covered in Edith's writeup on sharing
% immutable data
\section{The Sharing Pool Pattern}
\label{sec:sharing-pools}
\index{Sharing Pool}

\marginpar{A \textbf{Sharing Pool} stores canonical instances of data values that
would otherwise be replicated in many objects.} It is very common for
applications to store many copies of the same data in multiple data structures.
This is especially a problem with strings. Heaps can often store the same string
dozens or even hundreds of times. The sharing pool pattern is a way to amortize
the memory cost of data over many uses of it that overlap in time. If they don't
overlap in time, then you might still find it beneficial to amortize the
construction time, but this is a different lifetime pattern. It is a time-space
tradeoff, rather than a space-reduction optimization; see the resource pool
pattern below.

The general shape of the solution lies in a \emph{canonicalizing map}, one that
maps a new data structure to a previously constructed data structure with the
same shape and data values. Once you have made the decision to extend the
lifetime of an object, of those canonical instances, you have introduced a
lifetime management problem.


\begin{example}{Duplicate Strings}
You application loads data from a file. This data contains a large number of
name-value maps that will be used frequently throughout program execution.
These maps represent configuration information. The names come from a small set
of 16 distinct names. The set of string values 
is unknown at development time, but is known to be small;
you know there aren't going to be many distinct values, but you are unwilling or
unable to nail them down at compile time. How can these maps be stored in a memory
efficient way?
\end{example}

Without any special effort, each instance of this kind of configuration map
would store the some subset of same 16 key strings. Furthermore, each map would
store duplicates of the values. The following code snippet has those two
aspects of duplication:

\begin{shortlisting}
Map<String,String> map = new HashMap<String,String>();
void handleNextEntry() {
   String key = getNextString();
   String value = getNextString();
   map.put(key, value);
}
\end{shortlisting} 

Java provides a built-in mechanism for sharing the contents of strings across
many string instances. By \emph{interning}\index{String interning} a Java
\class{String}, you ensure that the returned \class{String} will only have
distinct storage if it is a string value that hasn't been interned yet. You can
modify the first try as follows:

\begin{shortlisting}
void handleNextEntry() {
   String key = getNextString().intern();
   String value = getNextString().intern();
   map.put(key, value);
}
\end{shortlisting} 

It is possible to do even better, if you have the luxury of modifying both ends
of the communication channel, i.e. both the serialization and this
deserialization code. There are only 16 distinct names used in all instances of
this configuration map. This seems like a perfect case for an enumerated type. An
enumerated type can be used to represent strings as numbers at runtime. The only
place the strings are stored is in the string constant pool\index{Java's Constant
Pools}. Each class, when compiled, keeps a pool of the strings that are used by
code in that class. In this way, an enumerated type is an even more highly
optimized sharing pool than that provided by the interning mechanism. 

Enumerated types offer an additional opportunity for decreased memory bloat.
Since, in this example, the keys can be represented as an enumerated type, you
can use the \class{EnumMap} to store the mapping. It is over 3 times as space
efficient as a \class{HashMap}, consuming only 28 bytes per collection and 8
bytes per entry compared to 120 bytes per collection and 28 bytes per entry for a
\class{HashMap}.

\begin{shortlisting}
enum PropertyName = {...};
Map<PropertyName,String> map = new EnumMap<PropertyName,String>(PropertyName.class);
void handleNextEntry() {
   PropertyName key = getNextPropertyName();
   Object value = getNextString().intern();
   map.put(key, value);
}
\end{shortlisting} 
