\chapter{Trading Space for Time: Caches, Resource Pools and Thread-Local Stores}
\label{chapter:trading-space-for-time}

There are three important cases of time-space tradeoffs. The first covers
the situation where recomputing attributes, rather than storing them, is a better
choice. The next two cover situations where spending memory to extend the
lifetime of certain objects saves sufficient time to be worthwhile: caches and
resource pools.

\section{Tools: Soft References}

\callout{soft-reference-rule}{Soft Reference Rule}{
Soft references must always be over values, not keys. Otherwise, testing
equality of keys will trigger a use of the reference. This will extend the
lifetme of the value, even though the only use of the entry was in checking to
see if its key matches another.	}

\section{Caches}

If the data stored in a data structure is frequently and expensively recomputed
or refetched, and the data values are the same every time, then it is worthwhile
to cache the computation or data fetch. The expense of re-fetching data from
external data sources and recomputing the in-memory structure can often be
amortized, at the expense of stretching the lifetime of these data structures. A
good cache defers the time that an object will be reclaimed, as long as there is
sufficient space to handle the flux of temporary objects your application
creates.


\section{Resource Pools}
\label{sec:resource-pools}
\index{Resource Pool}

\index{Amortizing Costs}
A cache can amortize the cost, in time, of fetching or otherwise initializing the
data stored in an object. A sharing pool can amortize the cost, in space, of
storing the same data in many separate objects. In both cases, the data is the
important part of what is stored.

\marginpar{A \textbf{Resource Pool} is a set of interchangeable storage or
external connections that are expensive to construct.} There is a third case,
where one needs to amortize the cost of the allocations, rather than the cost of
initializing or fetching the data that is stored in this object. A resource pool
stores the result of the allocation, not the data. Therefore, the elements of a
resource pool are interchangeable, because it is the storage, not the values that
matter. It is important to note that, though the data values are not the
important part, the elements of the pool are objects, and are thus intended to
store data! A resource pool handles the interesting case where the data is
temporary, but you need, for performance reasons, the objects to live across many
uses. The protocol for using a resource pool then involves reservation, a period
of private use of the fields of the reserved object, followed by a return of that
object to the pool.

Resource pooling only makes sense if the allocations themselves are expensive.
There are several reasons why a Java object can be expensive to allocate.
Creating and zeroing a large array\index{Large Arrays} in each iteration of a
loop can bog down performance. Creating a new key object to determine whether an
value exists in a map can sometimes contribute a great deal to the load of
temporary objects.

\index{Connection Pools}
A more important example of the need for amortizing the time cost of allocation
comes when this Java object is a proxy for resources outside of Java. If your
application accesses a relational database through the JDBC\index{JDBC}
interface, you will experience the need for resource pooling. There are two kinds
of objects that serve as proxies for resources involving database access. First
are the connections to the database. In most operating systems, establishing a
network connection is an expensive proposition. It also involves reservation of
resoures in the database process. Second are the precompiled SQL statements that
your application uses. As with the connections, these involve setup cost, of the
compilation itself, as well as the reservation of memory resources, that the
database uses to cache certain information about the query.

\section{Avoiding Leaks When Optimizing for Time}
\subsection{Example: Session State}
\subsection{Example: A Cache-like Sharing Pool}

\section{Avoiding Contention: Thread-local Stores}







