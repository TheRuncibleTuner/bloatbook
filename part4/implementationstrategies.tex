\chapter{Implementation Practices for Sound Lifetime Management}

Typically, objects die soon after the point in time of their last use, once all
dominating references are removed or naturally go out of scope. Without any
optimizations, objects live and die according to this \emph{natural lifetime}
that was discussed in \autoref{sec:natural-lifetime}. Sometimes it is beneficial
to extend, or shorten, the natural lifetime of an object. For example, if every
request creates an object of the same type, with the same, or very similar,
fields, then you should consider caching or pooling a single instance of this
object. There are four important cases of time-space tradeoffs. The first covers
the situation where recomputing attributes, rather than storing them, is a better
choice. The next three cover situations where spending memory to extend the
lifetime of certain objects saves sufficient time to be worthwhile: caches,
sharing pools, and resource pools.


% managing a reference queue

\section{Safeguards}
\subsection{Single Strong Reference Principle}
\subsection{home base}

\section{}
\subsection{Annotations}

\section{Safety Valves}

\section{Managing Reference Queues}


