\chapter{Implementation Practices for Sound Lifetime Management}
\label{chapter:lifetime-implementation-strategies}

Typically, objects die soon after the point in time of their last use, once all
dominating references are removed or naturally go out of scope. For a great many
objects, the normal flow of method invocations results in local variables going
out of scope, which renders these objects reclaimable without any special effort
on your part. In the absence of memory leaks, and without any optimizations,
objects live and die according to this \emph{natural lifetime}, as discussed in
depth in \autoref{sec:natural-lifetime}. 

However, the built-in lifetime mechanisms, by relying on objects going out of
scope, are insufficient to implement the more complicated patterns.
Implementations of the correlated lifetime pattern, introduced in
\autoref{sec:correlated-lifetime-pattern}, are very prone to memory leaks. You
may need an object to survive for a period of time that is not bound to any one
method invocation, but rather to the lifetime of another object. The lifetime of
some objects are indeed correlated with an invocation, as in the case of objects
correlated with a phase or request, but even here there are difficulties.
Oftentimes, the invocation that marks the beginning of a request is in a part of
the code outside of your control, or is distant from the allocation site of the
objects that must go away when the request finishes. Implementations of the
time-space tradeoff pattern, introduced in
\autoref{sec:time-space-tradeoffs-pattern}, can be ineffective if they aren't
sized properly. They, too, can result in memory exhaustion, e.g. if a cache's key
misimplement equality, or if it is sized too large.

It is important to code according to practices that will assure that an object
dies when it should. The correlated lifetime and time-space tradeoff patterns are
the most difficult cases to get right, and so those most in need of rigorous
coding practices.

% managing a reference queue

\section{Lifetime Management Principles}

\paragraph{Avoid Weak/Soft-Strong Reference Cycles}

\paragraph{The Single Strong Reference Principle}	
\paragraph{Home Base}
\paragraph{Safety Valves}

%\section{Example Applications of Lifetime Practices}
\section{The Correlated Lifetime Patterns}

\subsection{Annotations}

\begin{example}{Timestamp Annotation}
How can you associate a timestamp with an object in a way that avoids memory
leaks and that scales well to a highly concurrent workload?
\end{example}

We can start with the following code:

\begin{shortlisting}
class TimestampAnnotation<T> {
	T t;
	long timestamp;
}
List annotations;
for (String string : inputList) {
	...
	annotations.add(new WeakReference(new Wrapper<String>(string)));
	...
}
\end{shortlisting}

Despite your use of \class{WeakReference}, you would find that neither the main
object (the strings), nor the annotations, would ever be collected. This code
has two memory leaks. One of the leaks is due to a
violation of the first principal of the use of weak references: the annotations
strongly reference the objects being annotated. It is not always this easy to
debug problems in using weak references. Your application will hold on to objects
that you didn't expect. Quite often, it is difficult to even know that there is a
problem in the first place! The application may behave normally, except that it
will consume more memory than necessary; if this extra memory consumption pushes
it over your maximum heap size, then your application will crash --- you will
know something is wrong, but diagnosing this type of problem, a memory
leak\index{Memory Leak}, is quite difficult. It is better to keep the
three principles of weak references in mind, and design in a way that avoids
memory leaks in the first place. Your annotations can be modified to use a
\class{WeakReference} to the main object:

\begin{shortlisting}
class TimestampAnnotation<T> {
	WeakReference<T> t; // annotation only weakly refs main object
	long timestamp;
	
	TimestampAnnotation(T t) {
		this.t = new WeakReference(t);
	}
}
\end{shortlisting}

In this case, the annotation has no normal references to the annotated object,
and so it obides by the first rule of weak references. If you remember from
\autoref{chapter:delegation}, the code can be improved further to avoid the cost
of delegation. This version of the annotation class extends
\class{WeakReference}:

\begin{shortlisting}
class TimestampAnnotation<T> extends WeakReference<T> {
	long timestamp;
	
	TimestampAnnotation(T t) {
		super(t);
	}
}
\end{shortlisting}

Unfortunately, both of these updated versions h?
%%%%%%%
% old version??
%You could store the annotations in a map that is keyed by the
%original object, say of type \class{T}:
%
%\begin{shortlisting}
%Map<T, Date> timestamps = new HashMap<T, Date>();
%
%void addTimestamp(T t) {
	%timestamps.put(t, new Date());
%}
%Date getTimestamp(T t) {
%	return timestamps.get(t);
%}
%\end{shortlisting}
%
%This solution will function correctly, but suffers from a \emph{memory
%leak}\index{Memory Leak}. As the application runs, it will consume greater
%amounts of Java heap, up until the point when the \jre runs out of heap space
%to allocate any more objects. This solution leaks memory, because the
%\code{timestamps} map introduces a reference to the main objects. When the
%garbage collector scans the heap to see which objects are still alive, the
%references in this map will be among those that keep the objects alive. The
%next chapter discusses these issues in more detail. An improved solution would
%use the \class{WeakHashMap} from the Java standard libraries. By replacing the
%initialization of the \code{timestamps} map, we have the same functionality as
%before, but no memory leak.
%
%\begin{shortlisting}
%Map<T, Date> timestamps = new WeakHashMap<T, Date>();
%\end{shortlisting}
%
%Note that this same situation can hold even if you are able to modify the class
%definition. A common scenario requires annotations on only a subset of all
%instances of a class. In this case, is it not worth paying the memory cost to
%have the ability to annotate every single instance. Therefore, this is another
%case where a solution of side annotations, stored in a \class{WeakHashMap},
%shines.
%%%%%%%%%


\subsection{Phase/Request-Scoped Objects}

\subsection{Listeners}

\callout{listener-rule}{Listener Rule}{
Listener queues should weakly reference the callback object. This obviates to
maintain and debug code that explicitly deregisters the callback hook from the
listener queue.}

\section{The Sharing Pool Pattern}
\label{sec:sharing-pools}
\index{Sharing Pool}

\marginpar{A \textbf{Sharing Pool} stores canonical instances of data values that
would otherwise be replicated in many objects.} It is very common for
applications to store many copies of the same data in multiple data structures.
This is especially a problem with strings. Heaps can often store the same string
dozens or even hundreds of times. The sharing pool pattern is a way to amortize
the memory cost of data over many uses of it that overlap in time. If they don't
overlap in time, then you might still find it beneficial to amortize the
construction time, but this is a different lifetime pattern. It is a time-space
tradeoff, rather than a space-reduction optimization; see the resource pool
pattern below.

The general shape of the solution lies in a \emph{canonicalizing map}, one that
maps a new data structure to a previously constructed data structure with the
same shape and data values. Once you have made the decision to extend the
lifetime of an object, of those canonical instances, you have introduced a
lifetime management problem.


\begin{example}{Duplicate Strings}
You application loads data from a file. This data contains a large number of
name-value maps that will be used frequently throughout program execution.
These maps represent configuration information. The names come from a small set
of 16 distinct names. The set of string values 
is unknown at development time, but is known to be small;
you know there aren't going to be many distinct values, but you are unwilling or
unable to nail them down at compile time. How can these maps be stored in a memory
efficient way?
\end{example}

Without any special effort, each instance of this kind of configuration map
would store the some subset of same 16 key strings. Furthermore, each map would
store duplicates of the values. The following code snippet has those two
aspects of duplication:

\begin{shortlisting}
Map<String,String> map = new HashMap<String,String>();
void handleNextEntry() {
   String key = getNextString();
   String value = getNextString();
   map.put(key, value);
}
\end{shortlisting} 

Java provides a built-in mechanism for sharing the contents of strings across
many string instances. By \emph{interning}\index{String interning} a Java
\class{String}, you ensure that the returned \class{String} will only have
distinct storage if it is a string value that hasn't been interned yet. You can
modify the first try as follows:

\begin{shortlisting}
void handleNextEntry() {
   String key = getNextString().intern();
   String value = getNextString().intern();
   map.put(key, value);
}
\end{shortlisting} 

It is possible to do even better, if you have the luxury of modifying both ends
of the communication channel, i.e. both the serialization and this
deserialization code. There are only 16 distinct names used in all instances of
this configuration map. This seems like a perfect case for an enumerated type. An
enumerated type can be used to represent strings as numbers at runtime. The only
place the strings are stored is in the string constant pool\index{Java's Constant
Pools}. Each class, when compiled, keeps a pool of the strings that are used by
code in that class. In this way, an enumerated type is an even more highly
optimized sharing pool than that provided by the interning mechanism. 

Enumerated types offer an additional opportunity for decreased memory bloat.
Since, in this example, the keys can be represented as an enumerated type, you
can use the \class{EnumMap} to store the mapping. It is over 3 times as space
efficient as a \class{HashMap}, consuming only 28 bytes per collection and 8
bytes per entry compared to 120 bytes per collection and 28 bytes per entry for a
\class{HashMap}.

\begin{shortlisting}
enum PropertyName = {...};
Map<PropertyName,String> map = new EnumMap<PropertyName,String>(PropertyName.class);
void handleNextEntry() {
   PropertyName key = getNextPropertyName();
   Object value = getNextString().intern();
   map.put(key, value);
}
\end{shortlisting} 

There is an important variant of a sharing pool called the Bulk Sharing Pool.
Like a normal sharing pool, the goal of a bulk sharing pool is to amortize the
memor costs of storing data. However, rather than mitigate the costs of data
duplication, a bulk sharing pool aims to amortize the costs of Java object
headers across the elements in a pool. This is a topic that stretches notions of
how to store data beyond the normal Java box, and so will be discussed, along
with many similar matters, in \autoref{chapter:large-long-lived}.


\subsection{Annotation Pool}

\section{The Time-space Tradeoff Patterns}
There are four important cases of time-space tradeoffs. The first covers
the situation where recomputing attributes, rather than storing them, is a better
choice. The next three cover situations where spending memory to extend the
lifetime of certain objects saves sufficient time to be worthwhile: caches,
sharing pools, and resource pools.

\paragraph{Caches}

If the data stored in a data structure is frequently and expensively recomputed
or refetched, and the data values are the same every time, then it is worthwhile
to cache the computation or data fetch. The expense of re-fetching data from
external data sources and recomputing the in-memory structure can often be
amortized, at the expense of stretching the lifetime of these data structures. A
good cache defers the time that an object will be reclaimed, as long as there is
sufficient space to handle the flux of temporary objects your application
creates.


\callout{soft-reference-rule}{Soft Reference Rule}{
Soft references must always be over values, not keys. Otherwise, testing
equality of keys will trigger a use of the reference. This will extend the
lifetme of the value, even though the only use of the entry was in checking to
see if its key matches another.	}



\paragraph{Resource Pools}
\label{sec:resource-pools}
\index{Resource Pool}

\index{Amortizing Costs}
A cache can amortize the cost, in time, of fetching or otherwise initializing the
data stored in an object. A sharing pool can amortize the cost, in space, of
storing the same data in many separate objects. In both cases, the data is the
important part of what is stored.

\marginpar{A \textbf{Resource Pool} is a set of interchangeable storage or
external connections that are expensive to construct.} There is a third case,
where one needs to amortize the cost of the allocations, rather than the cost of
initializing or fetching the data that is stored in this object. A resource pool
stores the result of the allocation, not the data. Therefore, the elements of a
resource pool are interchangeable, because it is the storage, not the values that
matter. It is important to note that, though the data values are not the
important part, the elements of the pool are objects, and are thus intended to
store data! A resource pool handles the interesting case where the data is
temporary, but you need, for performance reasons, the objects to live across many
uses. The protocol for using a resource pool then involves reservation, a period
of private use of the fields of the reserved object, followed by a return of that
object to the pool.

Resource pooling only makes sense if the allocations themselves are expensive.
There are several reasons why a Java object can be expensive to allocate.
Creating and zeroing a large array\index{Large Arrays} in each iteration of a
loop can bog down performance. Creating a new key object to determine whether an
value exists in a map can sometimes contribute a great deal to the load of
temporary objects.

\index{Connection Pools}
A more important example of the need for amortizing the time cost of allocation
comes when this Java object is a proxy for resources outside of Java. If your
application accesses a relational database through the JDBC\index{JDBC}
interface, you will experience the need for resource pooling. There are two kinds
of objects that serve as proxies for resources involving database access. First
are the connections to the database. In most operating systems, establishing a
network connection is an expensive proposition. It also involves reservation of
resoures in the database process. Second are the precompiled SQL statements that
your application uses. As with the connections, these involve setup cost, of the
compilation itself, as well as the reservation of memory resources, that the
database uses to cache certain information about the query.





\section{Concurrency Issues}
\label{sec:lifetime-management-concurrency-issues}

If your program operates with many concurrent threads, you have to program
differently, because straightforward implementations of the above strategies will
result in concurrency issues. One of the primary problems will be lock
contention, as threads concurrently poll a reference queue. An important example
of this problem shows up in the implementation of a cache that can support many
concurrent users.\index{Caches, Concurrency Issues}

A cache is a map, usually of bounded size, with an eviction strategy for
maintaining that bound.\index{Caches} The Java standard library provides a
concurrent map implementation, in the form of the \class{ConcurrentHashMap}
class, but this is not a cache, because it has no eviction hooks with which one
can bound its size.

Following the soft reference rule, and using the basic guidelines for managing
reference queues from \autoref{sec:reference-queue-basics}, leads to a first
attempt at a \class{ConcurrentCache} implementation. You can extend the basic
concurrent hashmap, wrapping the map's values with soft references:
\begin{shortlisting}
class ConcurrentCache<K,V> extends ConcurrentHashMap<K,SoftReference<V>> {
   private final ReferenceQueue<V> refQueue = new ReferenceQueue<V>();
   
   protected void cleanupQueue() {
      SoftReference<V> v;
      while( (v = refQueue.poll()) != null) {
         remove(???); // oops!
      }
   }
}
\end{shortlisting}
Oops! This implementation provides no way to remove the key from the map, when
cleaning up the evicted entries. To fix this, you'll need to stash a pointer to
the key in the soft reference wrapper. It would be nice if the
\class{ConcurrentHashMap} implementation let you extend
its implementation so that its \class{\$Entry} class extended soft reference;
the \class{\$Entry} would serve this role perfectly. Instead, you have to
replicate this pointer structure, at a silly but unavoidable cost of memory
bloat. You can do so in a \class{CacheSoftReference} wrapper:
\begin{shortlisting}
class CacheSoftReference<K, V> extends SoftReference<V> {
   private final K k;
   
   public CacheSoftReference(K k, V v, ReferenceQueue<V> refQueue) {
      super(v, refQueue);
      this.k = k;
   }
}

class ConcurrentCache<K,V> extends ConcurrentHashMap<K,CacheSoftReference<K,V>> {
   private final ReferenceQueue<V> refQueue = new ReferenceQueue<V>();
   
   protected void cleanupQueue() {
      SoftReference<V> v;
      // poll causes lock contention!
      while( (v = refQueue.poll()) != null) {
         remove(v.k);
      }
   }
   
   public V put(K key, V value) {
      cleanupQueue();
      return super.put(key, new CacheSoftReference<K,V>(key,value,refQueue));
   }
   
   public V get(K key) {
      cleanupQueue();
      return super.get(key);
   }
}
\end{shortlisting}
This implementation still suffers from several critical problems. First, every
call to \code{put} must check the reference queue for pending evictions in order
to avoid unbounded growth of the eviction queue --- in the steady state,
\code{put} calls are likely to cause evictions. Even though \code{get} calls
won't cause evictions, in order to avoid pending evictions piling up as cached
elements are discovered to be unused, every call to \code{get} must also poll for
evictions. This can result in foiling the concurrency aspect of the
\class{ConcurrentHashMap}. Second, if the cache as a whole goes unused for a long
period of time, the pending evictions will pile up.

The only way to fix the lock contention problem, at least as of Java 6, is to
spawn a thread that periodically polls the reference queue for evicted entries.
This will also fix the second problem. This spawned thread's \code{run} method
will look just like the \code{cleanupQueue} method above, except that it should
loop indefinitely, and call \code{refQueue.remove()} rather than \code{poll()};
the former blocks until an eviction occurs (though you must still be careful to
check the return value for \code{null}, despite what the Javadocs claim).

At first sight, it would seem that you should be able to remove the calls to
\code{cleanupQueue} from the \code{put} and \code{get} methods. This, after all,
was the whole point of introducing the cleanup thread. However, this modified
implementation, while an improvement, suffers from a new problem. Now, if you
remove the \code{cleanupQueue} calls, when there is a large spike of \code{put}
calls in a short period of time, you are at risk of running out of Java heap due
to a large pileup of pending evictions.

You must have a safety valve in place to prevent this situation. One possibility
is to keep an approximate count of the number of \code{put} calls, and call
\code{cleanupQueue} only periodically. In order to avoid lock contention in
maintaining this count, you can do so in an unsynchronized way. There is still a
pathologic possibility that every racey increment of the put counter won't
actually increment the counter. If this worries you, you can use an
\class{AtomicInteger}, at increased expense. Instead of calling
\code{cleanupQueue} directly, the \code{put} method now calls a new
\code{helpCleaner} method:
\begin{shortlisting}
   static private final int SAFETY_VALVE = 1000;
   private void helpCleaner() {
      if (putCount.incrementAndGet() >= SAFETY_VALVE) {
         putCount.set(0);
         cleanupQueue();
      }
   }
\end{shortlisting}
There is no reason for \code{get} calls to call this method. The only point of
this safety valve is to avoid a sudden large influx of \code{put} calls. Indeed,
in this final implementation, the \code{ConcurrentCache} class needn't override
the \code{get} method of \code{ConcurrentHashMap}.
