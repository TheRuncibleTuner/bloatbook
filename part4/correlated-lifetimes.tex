\chapter{Correlated Lifetime}
\label{chapter:correlted-lifetimes}



There are important lifetime management policies that are not expressible via the
normal mechanisms. When referenced by a local variable, an object lives or dies
with the scope of the variable; when referenced by another object, it lives or
dies along with that object (both, of course, in the absence of overwriting a
reference). The Java specification provides three other mechanisms that let you
guide the \jre to the right time for an object to exit limbo: weak references,
soft references, and thread-local storage.

\section{Using Weak References}
\index{Weak References}

Java provides a low-level mechanism that one can use to implement a
correlated lifetime memory management policy, in the form of weak references.
The standard library exposes this feature in the class
\class{java.util.WeakReference}. Using this class correctly is
difficult, because the semantics of weak references does not directly map to
any important application-level use cases.

\begin{definition}
When calling the constructor \class{new WeakReference(obj)},
 this new instance will maintain a reference to \code{obj}, however
 \code{obj} will exit limbo, and become a candidate for cleanup processing by 
the \jre, as if that reference did not exist.
\end{definition} 

When used in this way, weak references don't keep an object alive longer than it
otherwise would have, in the absence of weak references. 
This seems pretty far from anything an application might
need. Still, you can use this low-level feature to implemented correlated
lifetime, as long as you're careful. 
When used to implement correlated lifetime policies, weak references may
indeed delay the time till an object exits limbo.
Improper use of a \class{WeakReference} will
render your code worse off than before. It is quite possible that you will not
have achieved the correlated lifetime that you need, but in a way that is hard
to tell. Even worse, when using weak references, you can introduce memory
leaks. Be very cautious when using them, and follow these rules.

\callout{weaks}{Rules for Using \class{java.util.WeakReference}}{
	In order to assure that you use of weak references works properly, you must
	follow three rules:
	\begin{itemize}
      \item Your instances of \class{WeakReference} must not, directly or
      indirectly, maintain a non-weak reference to the object you wish to
      annotate. It is best to maintain a collection of non-weak references to
      the annotated objects, and use a local variable, or a subclass of
      \class{WeakReference} for any other ways you refer to the annotated object.
      
      \item You must ensure that the \class{WeakReference} objects (or
      subclasses thereof) that you create will exit limbo no sooner than the
      annotated objects. Otherwise, these objects themselves, following the
      rules of \autoref{tab:limbo-exit}, will exit limbo too early. So, if the
      annotated objects are referenced by a collection that is in turn
      referenced by a static field, then the same must be true for the weak
      reference objects as well. 

      \item Since you have to maintain two, parallel, collections to maintain
      references to the annotated objects, and to the annotations, you must
      ensure that exit limbo in lockstep. It is best to create your instances
      of \class{WeakReference} with a \class{ReferenceQueue} parameter. You
      must periodically call \code{poll} on this queue, and remove the
      \class{WeakReference} instances from the parallel registry.
    \end{itemize}
}

You can see that, despite the benefit of some support from the \jre, there
is quite a bit of memory management that you are left with. Luckly, the
standard library ships with a \class{WeakHashMap}\index{WeakHashMap} which deals
with some, but not all, of the legwork of managing weak references. It will
handle the second and third items, but not the first. It is still up to you to
ensure that none of your annotations, directly or indirectly, reference the
annotated object. 

\begin{example}{Timestamp Annotation}
How can you associate a timestamp with an object in a way that avoids memory
leaks and that scales well to a highly concurrent workload?
\end{example}

We can start with the following code:

\begin{shortlisting}
class TimestampAnnotation<T> {
	T t;
	long timestamp;
}
List annotations;
for (String string : inputList) {
	...
	annotations.add(new WeakReference(new Wrapper<String>(string)));
	...
}
\end{shortlisting}

Despite your use of \class{WeakReference}, you would find that neither the main
object (the strings), nor the annotations, would ever be collected. This code
has two memory leaks. One of the leaks is due to a
violation of the first principal of the use of weak references: the annotations
strongly reference the objects being annotated. It is not always this easy to
debug problems in using weak references. Your application will hold on to objects
that you didn't expect. Quite often, it is difficult to even know that there is a
problem in the first place! The application may behave normally, except that it
will consume more memory than necessary; if this extra memory consumption pushes
it over your maximum heap size, then your application will crash --- you will
know something is wrong, but diagnosing this type of problem, a memory
leak\index{Memory Leak}, is quite difficult. It is better to keep the
three principles of weak references in mind, and design in a way that avoids
memory leaks in the first place. Your annotations can be modified to use a
\class{WeakReference} to the main object:

\begin{shortlisting}
class TimestampAnnotation<T> {
	WeakReference<T> t; // annotation only weakly refs main object
	long timestamp;
	
	TimestampAnnotation(T t) {
		this.t = new WeakReference(t);
	}
}
\end{shortlisting}

In this case, the annotation has no normal references to the annotated object,
and so it obides by the first rule of weak references. If you remember from
\autoref{chapter:delegation}, the code can be improved further to avoid the cost
of delegation. This version of the annotation class extends
\class{WeakReference}:

\begin{shortlisting}
class TimestampAnnotation<T> extends WeakReference<T> {
	long timestamp;
	
	TimestampAnnotation(T t) {
		super(t);
	}
}
\end{shortlisting}

Unfortunately, both of these updated versions h



\section{Using Finalizers}


\section{Correlation Gone Wrong: Memory Leaks}
\index{Memory Leaks}


talk about string interning, archetest2 (which was interning something bigger),
weak versus soft versus strong
