%basics of MM: when gc runs, Xmx and when heap expands, what reclaimed what not,
%permspace, finalizers/phantom refs, class unloading and lifetime of statics,
%object lifecycle. maybe introduce weaks and softs


\chapter{The Object Lifecycle}
%Every object created by your application lives for an interval of time from its
%creation to the point that the Java runtime gets around to collecting it. An object's {\em natural} lifetime is defined by the
%interval of time between its first and last necessary use. %cite drag paper
%here?


In a \emph{well-behaved} application, an object's lifetime spans its allocation,
use, and the short period during which the \jre takes control and reclaims the
space. For some subset of an object's actual lifetime, that is the time from
creation to reclamation, your application will make use of the data stored in its
fields. \autoref{fig:typical-lifecycle} illustrates the lifecycle of a typical
object in a well behaved application.

\begin{figure}
	\includegraphics[width=0.9\textwidth]{part4/Figures/lifetime/object-lifecycle}
	\caption{Timeline of the life of an object.}
	\label{fig:typical-lifecycle}
\end{figure}

\begin{example}{Parsing a Date} Consider a loop that shows an easy way to parse
a list of dates. What objects are created, and what are their lifetimes?
\begin{shortlisting}
for (String string : inputList) {
	ParsePosition pos = new ParsePosition(0);
	SimpleDateFormat parser = new SimpleDateFormat();
	System.out.println(parser.parse(string, pos));
}
\end{shortlisting}
\end{example}

For each iteration of this loop, this code takes a date that is represented as a
string and produces a standard Java \class{Date} object. In doing so, a number of
objects are created. Two of these are easy to see, in the two \code{new} calls
that create the parse position and date parser objects. The programmer who wrote
this created two objects, but many more are created by the standard libraries to
finish the task. These include a calendar object, number of arrays, and the
\class{Date} itself. None of these objects are used beyond the iteration of the
loop in which they were created. Within one iteration, they are created, almost
immediately used, and then enter a state of \emph{limbo}.

\callout{limbo}{Objects in Limbo}{
\index{Limbo}
In limbo, an object will never be used again, or at least not for long time,
but the \jre doesn't yet know that this is the case. The object hangs
around, taking up space in the Java heap until the point when it exits limbo.}

The \code{pos} object represents to the parser the position within the
input string to begin parsing. The implementation of the \code{parse} method
uses it early on in the process of parsing. Despite being unused for the
remainder of the parsing, the \jre does not know this until the current
iteration of the loop has finished. This time in limbo also includes the
entirety of the call to \code{System.out.println}, an operation entirely
unrelated to the creation or use of the parse position object. Once the current
loop iteration finishes, these two objects will exit limbo, and become garbage
collectible.\index{Exiting Limbo}

\begin{figure}
	\centering
	\includegraphics[width=\textwidth]{part4/Figures/lifetime/states}
	\caption{After its last use, an object enters a kind of limbo: the application
	is done with it, but the \jre hasn't yet inferred this to be the case. When an
	object exits limbo depends on the way it is referenced.}
		\label{fig:limbo-exit}
\end{figure}


\begin{comment}
\begin{figure}
	\centering
	\subfigure[The lifecycle of a typical object and its data.]{
	\label{fig:typical-lifecycle1}
			\includegraphics[width=0.95\textwidth]{part4/Figures/lifetime/object-lifecycle}
	}
	\subfigure[A situation where there are long periods between uses of an
	object's data.]{
	\label{fig:typical-lifecycle2}
		\includegraphics[width=0.95\textwidth]{part4/Figures/lifetime/object-lifecycle-lulls}
	}
	\subfigure[The lifecycle of the data  that is loaded from
	disk three times, and the objects that store it.]{
	\label{fig:typical-lifecycle3}
		\includegraphics[width=0.9\textwidth]{part4/Figures/object-lifecycle2}
	}
	\caption{Examples of Natural and Actual lifetimes.}
	\label{fig:typical-lifecycle}
\end{figure}
\end{comment}


