\chapter{A Comparison of Sizings on \jres}
\label{chapter:jre-comparison}

TODO:  Let's make a clear separation of information readers need to estimate the
size of their applications vs. the maximum address space issues.  Otherwise the
tables will be too large. That will also make the text more
problem-oriented.  We need three tables:
\begin{itemize}
\item Object-level overheads: (32 vs. 64 vs. compressed) x (Sun vs. IBM),
showing (object header size, array header size, object alignment size, bytes per
reference) 
\item Field-level overheads, in two parts.
\begin{itemize}
  \item a. (type) x (arch) show field alignment
  \item b. Show the field-packing rules for each each JRE (may be a bullet list
rather than a table)
\end{itemize}
\item Addressable bytes: (Sun vs. IBM) x (OS), showing bytes.  Some slots will
be blank.
\end{itemize}

The majority of the body of this book focuses on the Sun 32-bit \jre. This
appendix provides supplementary material regarding the sizing criteria used by
other \jres, and how various configuration choices affect these parameters.

\section{Sizing Criteria}


\begin{table}
\centering
\begin{tabular}{lcccc}
\toprule
Platform & \shortstack{Bytes per\\Reference} & \shortstack{Bytes per\\Header} & Alignment & \shortstack{Addressible\\Bytes} \\
\cmidrule(r){1-1} \cmidrule(r){2-5}
IBM zOS 31-bit &  4 & 12    & 8 & 1.3GB \\ 
Windows 32-bit &  4 & 8--12 & 8 & 1.8GB \\
AIX 32-bit     &  4 & 12    & 8 & 2.5-3.25GB\\ 
Solaris 32-bit &  4 & 8     & 8 & 3.5GB \\ 
\cmidrule(r){1-1} \cmidrule(r){2-5}
Linux 64-bit   &  8 & 24    & 16 & 128TB \\
AIX 64-bit     &  8 & 24    & 16 & $>$ 1PB \\
compressed refs & 4 & 12 & 8 & 28-32GB \\
extended compressed refs & 4 & 12 & 16 & 64GB \\
\bottomrule
\end{tabular}
\caption{Sizing information for various platforms.}
\label{tab:sizing-criteria}
\end{table}

\autoref{tab:sizing-criteria} summarizes the important sizes that \jres use.
These include the number of bytes each reference consumes, the number of bytes
each object header consumes, the boundary (in bytes) to which each object
allocation is aligned, and the maximum number of bytes a Java application can
address. These values vary fairly widely, from one \jre to another. 

The value that varies most is the maximum heap size a Java application can use.
If you aren't running on a 64-bit \jre, you should exercise extreme caution in
setting the heap size too high. For example, on a 32-bit Windows platform, if
you request a heap size of 1.8GB, then your application will only have a few
hundred megabytes  of address space left over for native resources. All of your
byte code, compiled code, and all of the \jres metadata about classes and the
heap, need to fit into this small window. On AIX, if you request 3.25GB of Java
heap, the story is similar.

\begin{wraptable}{r}{0.5\textwidth}
\centering
\begin{tabular}{ll}
\toprule
Provider & Command-line Argument \\
\cmidrule(r){1-1} \cmidrule(r){2-2}
IBM & -Xcompressedrefs \\
Sun & -XX:+UseCompressedOops \\
JRockit & -XXcompressedrefs \\
\bottomrule
\end{tabular}
\caption{Options for specifying that you wish the \jre to use compressed references. This is only relevant for 64-bit \jres.}
\end{wraptable}

\section{Compressed references}
\index{Compressed References}

TODO: This duplicates discussion we have in the Objects and Delagation Chapter.
- let's move info that we really need to that chapter, unless it's very
detailed.  Let's keep the Appendix mainly for reference details.


 One potential downside to switching to a 64-bit \jre is an increase
in memory consumption. Each reference will consume 8 bytes, and each object header will
consume as much as 24 bytes. Luckily, most \jres support \emph{compressed
references}. When using compressed references, references and object headers
consume as much as they would on a 32-bit platform, which is nice. There are two
downsides. One is that the maximum heap size is far lower than it would be with
full 64-bit pointers, though still higher than with a 32-bit \jre. For example,
the Sun and IBM \javasix \jres support heap sizes of only up to 32GB; in some
cases, the limit is 25 or 28GB. The other downside is a potential, though small,
decrease in performance. THe hardware expects 64-bit pointers, but the heap only
stores 32 bits. Therefore, each memory operation may require an extra shift
operation. Some \jres are clever enough to avoid this expense, if the requested
maximum heap size is less than 4GB.

For some \jres, if you are willing to pay a higher memory overhead, then you can
support up to 64GB of memory, while still using only 32-bit references. When
\jres operate in this mode, each object is aligned to a 16 byte boundary. This
may result in an increase alignment overhead, compared to the normal 8-byte
alignment. For example, say a \class{Double} boxed scalar has 8 bytes of data
and a 12-byte header. This adds up to 20 bytes. With 8-byte alignment, you will
pay 4 bytes for alignment cost. With 16-byte alignment, the alignment overhead
is 12 bytes. On the other hand, for a \class{String} or \class{Integer}, the
alignment costs remain unchanged, whether the \jre aligns to 8- or 16-byte
boundaries.

Many \jres will automatically enable this support for compressed references. You
should consult the documentation of your specific \jre provider to know whether
this is the case.\footnote{On \javaseven, 64-bit
Sun \jres have this enabled by default, as long as your heap size is smaller
than 32 gigabytes.}

%Address compression is available in the Sun \javasix (Update 14) release,
%enabled with the option -XX:+UseCompressedOops.

%It is available in IBM \javasix (Service Release 4) with the
%option -Xcompressedrefs.

\section{Identity Hashcode in Object Headers}

Depending on the \jre, the object header may actually be 4 bytes smaller than
indicated in \autoref{tab:sizing-criteria}. Some \jres leave room for the
identity hashcode in the header of every object. The identity hashcode is
usually the address of the object. If the garbage collector decides to move the
object to another address at some point, then it has to store the original
address somewhere. This is necessary so that the \code{identityHashcode} method
always returns the same value, for the lifetme of a given object. The values in
the table assume that the \jre preallocates space for this identity hashcode,
and so every object pays the expense, even if the object's identity hashcode is
never used. Some \jres are clever enough to store the identity hashcode only
when it is needed: if it is used, and, even then, only if the object is moved
from its original location. The only way you can know whether this is the case
is to write a small program that allocates a large number of \class{Integer}
objects. You know how much space you expect the object to consume, based on the
4 bytes of actual data, and the expected header size. Try it, and see how many
you can fit, in a given size of heap.

