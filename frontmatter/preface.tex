\chapter*{Preface}
\label{chapter:preface}

Over the past ten years, we have worked with developers, testers, and
performance analysts to help fix memory-related problems in large Java
applications.  During the many years we have spent studying the performance of
Java applications, it has become clear to us that the problems related to memory
is a topic worthy of a book. In spite of the fact that computers are now
equipped with gigabytes of memory, Java developers face an uphill battle getting
their applications to fit in memory. Ten years ago, a 500MB heap was considered
big. Now it is not unusual to see applications that are having trouble fitting
into 2 to 3 gigabyte heaps, and developers are turning to 64-bit architectures
to be able to run with even larger heaps.

Java heaps are not just big, but are often bloated, with as much as 80\% of
memory devoted to overhead rather than "`real data"'. This much bloat is an
indication that a lot of memory is being used to accomplish little. We have seen
applications where a simple transaction needs 500K for the session state for one
user, or 1 Gigabyte of memory to support only a few hundred users.

By the time we are called in to help with a performance problem, the situation
is often critical. The application is either in the final stages of testing or
about to be deployed. Fixing problems this late in the cycle is very expensive,
and can sometimes require major code refactoring. It would certainly be better
if it were possible to deal with memory issues earlier on, during development or
even design.

Why ...? Java developers face some unique challenges when it comes to memory.
First, much is hidden from view. A Java developer who assembles a system out of
reusable libraries and frameworks is truly faced with an "`iceberg"', where a
single call or constructor may invoke many layers of hidden framework code. We
have seen over and over again how easy it is for space costs to pile across the
layers. %Framework designers have an especially hard problem in addition, of
trying to predict their usage, and design for every possible case, often an
impossible task, usually leading to waste for some case that matters to you!
While there is much good advice on how to code flexible and maintainable
systems, there is little information available on space. The space costs of
basic Java features and higher-level frameworks can be difficult to ascertain.
In part this is by design - the Java programmer has been encouraged not to think
about physical storage, and instead to let the runtime worry about it. The lack
of awareness of space costs, even among many experienced developers, was a key
motivation for writing this book. By raising awareness of the costs of common
programming idioms, we aim to help developers make informed tradeoffs, and to
make them earlier.

The design of the Java language and standard libraries can also make it more
difficult for programmers to use space efficiently.  Java's data modeling
features and managed runtime give developers fewer options than a language like
C++, that allows more direct control over storage. Taking these features out of
the hands of developers has been a huge plus for ease of learning and for safety
of the language. However, it leaves the developer who wants to engineer frugally
with fewer options. %more important to make these choices carefully Helping
developers make informed decisions between competing options was another aim for
the book.


%lifetime management. awareness of mechanisms & importance of understanding mechanisms.


%Compared to systems languages like C, Java space costs are high, even for the most basic building blocks. This makes it all the more important for developers to be aware of memory costs.  Java heaps are not just big, but are often bloated, with as much as 80\% of memory devoted to overhead. Memory bloat can have a serious impact on development schedules and on the scalability of deployed systems. Fortunately, bloated designs are not an inevitable consequence of object-oriented development.

This book is a comprehensive, practical guide to memory-conscious programming in
Java. It addresses two different and equally important aspects of using memory
well. Much of the book covers how to represent your data efficiently. It takes
you through common modeling patterns, and highlights their costs and discusses
tradeoffs that can be made. The book also devotes substantial space to managing
the lifetime of objects, from very short-lived temporaries to longer-lived
structures such as caches. Lifetime management issues are a common source of
bugs, such as memory leaks, and inefficiency (mostly performance).  Throughout
the book we use examples to illustrate common idioms. Most of the examples are
distilled from more complex examples we have seen in real-world applications. 
Throughout the book are also guides to Java mechanisms that are relevant to a
given topic.  These include features in the language proper, as well as the
garbage collector and the standard libraries.

While the book is a collection of advice on practical topics, it is also
organized so as to give a systematic approach to memory issues. When read as a
whole it can be helpful in seeing the range of topics that need to be
considered, especially early in design. That does not mean that one must read
the whole book in order, or do a comprehensive analysis of every data structure
in your design, in order to get the benefit. The chapters are written to stand
on their own where possible, so that if a particular pattern comes up in your
code you can quickly get some ideas on costs and alternatives. At the same time,
familiarizing yourself with a few concepts in the Introduction will make the
reading much easier.



% This book is a practical, systematic, and comprehensive guide to
% memory-conscious programming in Java. It walks though numerous examples taken
% from real applications, illustrating common design problems that lead to
% memory bloat, and looks inside the Java collection classes and runtime. It
% details a methodology for programmers to follow. Our aim is to empower
% developers to avoid pitfalls, make informed tradeoffs, and to show how
% dramatic improvements in memory efficiency are sometimes possible with a
% little care.

The book is appropriate for Java developers (experienced and novice alike),
especially framework and applications developers, who are faced with decisions
every day that will have impact down the line in system test and production. It
is also aimed at technical managers and testers, who need to make sure that Java
software meets its performance requirements.  This material should be of
interest to students and teachers of software engineering, who would like to
gain a better understanding of memory usage patterns in real-world Java
applications. Basic knowledge of Java is assumed.

Much of the content relies on knowing or measuring the size of objects at
runtime. Sizes vary depending which JRE you are using. Our reference JRE is Sun
\javasix Update 14. Unless otherwise stated, all sizes are for this reference
JRE. The book is self-contained in that it teaches how to calculate object sizes
from scratch. We realize, of course, that this can be a tedious endevour, and so
the appendix provides a list of tools and resources that can help with memory
analysis. Nevertheless, we belief that performing detailed calculations are
pedagogically important.

[NOTE(GSS): Roadmap could be in Intro, possibly interwoven with approach.]

The book is divided into four parts:

Part 1 introduces an important theme that runs through the book: the health of a
data design is the fraction of memory devoted to actual data vs. various kinds
of infrastructure. In addition to size, memory health can be helpful for gauging
the appropriateness of a design choice, and for comparing alternatives. It can
also be a powerful tool for recognizing scaling problems early.

Part 2 covers the choices developers face when creating their physical data
models, such as whether to delegate data to separate classes, whether to
introduce subclasses, and how to represent sparse data and relationships. These
choices are looked at from a memory cost perspective. This section also covers
how the JVM manages objects and its cost implications for different designs.
  
Part 3 is devoted to collections. Collection choices are at the heart of the
ability of large data structures to scale. This section covers, through
examples, various design choices that can be made based on data usage patterns
(e.g. load vs. access), properties of the data (e.g. sparseness, degree of
fan-out), context (e.g. nested structures) and constraints (e.g. uniqueness). 
We look closely at the Java collection classes, their cost in different
situations, and some of their undocumented assumptions. We also look at some
alternatives to the Java collection classes.
 
Part 4 covers the topic of lifetime management, a common source of inefficiency,
as well as bugs. This section examines the costs of both short-lived temporaries
and long-lived structures, such as caches and pools.  We explain the Java
mechanisms available for managing object lifetime, such as ThreadLocal storage,
weak and soft references, and the basic workings of the garbage collector.
Finally, we present techniques for avoiding common errors such as memory leaks
and drag. %plus making things fit.

