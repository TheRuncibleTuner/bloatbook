\chapter{Why Memory Costs are Important}

If you are reading this book, you may already know that it is not unusual for
Java programs to run out of memory. If you do think that Java memory costs is
not an important design issue, hopefully this chapter will convince you
otherwise. This is a topic that requires counting bytes, which may seem like a
strange activity for a Java programmer, accustomed to rapid assembly of
applications from assorted libraries. At its core, programming is an engineering
discipline, and there is no escaping the fact that the consumption of any finite
resource must be measured and managed.

\section{Quiz}

To start you thinking about counting bytes, here is a quiz to test how good you
are at estimating sizes of Java objects. Assume a 32-bit JVM.
\begin{verbatim}

   Question 1: What is the size ratio in bytes Integer to int?
   
      a. 1:1
      b. 1.33:1
      c. 2:1
      d. 4:1
      e. 8:1
   
   Question 2: How many bytes in an 8-character string?

      a. 8 bytes
      b. 16 bytes
      c. 20 bytes
      d. 40 bytes
      e. 56 bytes
 
   
   Question 3: Which statement is true about a HashSet compared to 
               a HashMap with the same number of entries?
               
      a. It has fewer data fields and less overhead.
      b. It has the same number of data fields and more overhead.
      c. It has the same number of data fields and the same overhead.
      d. It has more data fields and it has less overhead.
                  
                       
   Question 4: Arrange the following 2-element collections in size order:
    
      ArrayList, HashSet, LinkedList, HashMap
          
   Question 5: How many collections are there in a typical heap?
   
      a. between five and ten
      b. tens
      c. hundreds
      d. thousands
      e. one or more orders of magnitude bigger than above

   Question 6: What is the size of an empty ConcurrentHashMap?
   (Extra Credit)
      a. 17 bytes
      b. 170 bytes
      c. 1700 bytes
      d. 17000 bytes
      e. 500 bytes
           

ANSWERS: 1d, 2e, 3b, 
         4 ArrayList LinkedList HashMap HashSet, 
         5e, 6c                 
\end{verbatim}

If you look inside a typical Java heap, it is mostly filled with the kinds of
objects used in the quiz --- boxed scalars, strings, and collections. Every time
a program instantiates a class, there is an object created in the heap, and as
shown by the 4:1 size ratio of \texttt{Integer} to \texttt{int}, objects are not
cheap.

Strings often consume half of the heap and are surprisingly costly. If you are a
C programmer, you might think that an 8-character string should consume 9 bytes
of memory, 8 bytes for characters and 1 byte to indicate the end of the string.
What could possibly be taking up 56 bytes? Part of the cost is because Java uses
th 16-bit Unicode character set, but this accounts for only 16 of the 56 bytes.
The rest is various kinds of overhead.

After strings, collections are the most common types of objects in the heap. In
typical real programs, having 100's of thousands and even millions of collection
instances in a heap is not at all unusual. If there are a million collections in
the heap, then the collection choice matters. One collection type might use 20
bytes more than another, which may seem insignificant, but in a production
execution, the wrong choice can add close to 20 megabytes.

\texttt{ConcurrentHashMap}, compared to the more common collections in the
standard library, is surprisingly expensive. If you are used to creating
hundreds of \texttt{HashMaps}, then you might think that it is not a problem to
create hundreds of \texttt{ConcurrentHashMaps}. There is certainly nothing in
the API to warn you that \texttt{HashMaps} and \texttt{ConcurrentHashMaps} are
completely different when it comes to memory usage. This example shows why
understanding memory costs are important.

The quiz gives some sense of how surprising the sizes are for the Java basic
objects, like \texttt{Integers}, strings, and collections. When code is layered
with multiple abstractions, memory costs become more and more difficult to
predict.


\section{Magnitude of the Problem}

Over the past ten years, we have worked with developers, testers, and
performance analysts to help fix memory-related problems in large Java
applications. Typically, there is an out-of-memory error, which ends up being
either a memory leak or a scalability problem. A memory leak is a common, nasty
kind of lifetime management bug that can be extremely hard to track down. A
scalability problem, on the other hand, is not really a bug, but a design flaw.
The application by design is too big to fit into memory. For example a web
application may not scale to the required number of users because the size of a
single user session is too big. Fixing a scalability problem may require
extensive code refactoring.
  
Memory-related problems often show up late in the development cycle, during load
testing or even after deployment, when fixing the problem can be very costly.
When you are about to go into production, you do not want to discover that you
can only support a few hundred users, especially when the heap is already a few
gigabytes. This book provides tools and techniques to help avoid memory-related
problems early on.

We believe that the problems are getting worse, and the need for these tools and
techniques is becoming more and more pressing. First, Java heap sizes have
steadily been increasing.  Ten years ago, a 500MB heap was considered big. Now
it is not unusual to see applications that are having trouble fitting into 2 to
3 gigabyte heaps, and developers are looking at 64-bit architectures to be able
to run with even larger heaps.  These huge heaps are particulary alarming when
compared to the task at hand. For example, it is hard to justify a simple
transaction using 500K session states, which we have seen in a real application.

Secondly, the technology landscape is changing. While processor speeds have been
doubling every two years following Moore's law, a physical limit has now been
reached preventing further processor speedup. Instead, the number of processor
cores on a chip is expected double every two years. However, memory bandwidth
and cache sizes will not grow proportionally. Larger heaps combined with
relatively less bandwidth is a recipe for a big performance hit going forward.
Additionally, the rapidly growing number of small embedded processors requires
much more efficient use of memory.

Lastly, there is the green argument. If your application is not scaling, you may
need to add more servers, which in turn can increase your power consumption and
electric bill. This is a big problem for ever-growing data centers.


2.4 The Iceberg Effect

The one comon thread is that it is incredibly easy to build applications in Java
that use a lot of memory,

So what's going on? What's not going on is that stupid programmers are writing
stupid code; that's not what's happening.  And sometimes we hear, oh, that must
just be bad programmers, but in fact, we have very good programmers writing this
kind of code for many, many good reasons. A lot of the problems we look at are
mistakes made, or just where people couldn't know the cost, or they haven't
considered cost, and are just not analyzing until it's too late.

So first of all, just the number of abstractions people are dealing with, and
that's one of the basic problems here, is that, people are assembling code now,
rather than building it, and this is something that we've been dreaming of for
years, that people take reusable components off the shelf, and they glue them
together, and they have a system. And in fact the framework designers are doing
that themselves.  We call it the iceberg effect, where I code something, I write
some application code, it calls some framework, or a bunch of frameworks, and
hidden underneath here are is sort of the all the programmer sees is the tip of
the iceberg, and the consequences are completely hidden, in terms of what the
costs are, and that is something that people have been encouraged to be
programming in this style. They shouldn't be thinking about performance,
certainly not in terms of letting it mess up their good design, and even when
they do want to think about performance, sometimes it's very, very hard to look
inside these things.  Especially, at the risk of mixing metaphors here, the
framework developers are dealing with the same issues themselves, where
frameworks themselves are assembling code as well from other frameworks, and so
on down the line, eventually from libraries. In fact, as we will see in some of
our examples, inside the Java library themselves, the low level libraries are
calling other low level libraries.  String buffer uses ;;  Hashset interms of
hashmap, and so forth.  This is true for memory and performance.

2.5 Common myths about memory

In addition to the technical reality of dealing with all of this assembly, there
are some myths out there that are standing in the way.  One set of myths says
that things are fine, everything is cheap; there was an article on developer
works, called "Go ahead, make a mess",  and this idea that objects are free, at
least temporaries are, and everything will be taken care of for you. The JIT is
going to clean things up, the garbage collector will elp, all of this great
research on garbage collection and jit optimization, and you shouldn't have to
worry about that; just code whatever the best design is from a maintainability
standpoint, or whatever other standpoint, and the performance will magically be
taken care of.

In fact, in the memory space, the JIT is doing, all of the commercial JITs that
we know of are doing absolutely nothing in terms of storage optimization, so
that every single object, with all of its fields, ends up as an object in the
heap, taking up space.  We'll look into the detail of what that means. And
similarly the garbage collector, yes it is cleaning up temporary objects. But
even for temporaries there are other costs.

The construction of the objects in the first place; garbage collectors are only
dealing with shortlived objects, and footprint problems are a problem of
long-lived objects, which garbage collectors are not addressing at all.

Another common myth is that the frameworks must already be optimized, because
these were written by experts, and I cant tell you how often when we are
analyzing a trace, where people say, "Oh, I thought that it would have been
optimized, that's why I used it." And oftentimes attention wasn't given to
footprint. There has been very, very little attention to optimizations for
footprint.  When optimizations are done it's usually done for speed, certainly
in the standard libraries, that's been the case.

Also, even if people who wrote the frameworks did try to pay attention to
footprint, they often have a hard time predicting what the usage of their
framework is going to be. Sometimes they guess wrong, sometimes don't know where
to start.  So it's very unlikely that the framework has been optimized for your
particular use case.

Many developers know things are bad, but not how had, and we see this over and
over again. In fact, I just came back, about a month ago, working with a group
of IBM rational developers in Ottowa, a very strong group of people, very good
engineers, who were quite aware of how memory contrained they were, and even
they were surprised still when we looked at the actual cost, to see just how
costly things were. They were even more than they thought.

Then there are the people who just give up - I know Java is expensive, it's
always expensive, there is nothing I can do about it, it's just a cost of object
oriente programming in Java, and if I try to do anything, it's going to break my
good design. And so, hopefully, one of the things we want to achieve is to raise
awareness of cost. So that it's not a lost cause, there is some hope.  It's
possible to make informed tradeoffs, can't fix all problems. Identify places
where good engineering can help by the developer, and then where people are
going to hit a wall.

So there are lots and lots of problems to solve here. This is an area that
hasn't been addressed much at all.  There are just a few little pockets of
research. Will stimulate some nice ideas. I really encourage you to interrupt
and start discussion.

Finally, this is an issue for performance, not just for memory. There's a false
dichotomy in a lot of people's minds that say well, if you have a lot of
footprint it must be buying you something in terms of performance. But in
reality, that is not always the case. In fact, sometimes bad memory usage will
result in poor performance as well, even something as simple as I am using so
much of my heap for long-lived objects, then I have very little headroom for
temporaries.  And so my garbage collector has to run much more often.

Or I don't have enough room to size the caches for things I get from the
database as large as I like, so I go back to the database more than I like, so
this can have a huge performance cost. So these problems are very interrelated.