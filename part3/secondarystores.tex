\chapter{When It Won't Fit: Working with Secondary Stores}

\section{Serialization}
\index{Serialization}

There are often situations where your data structures must be shuttled in and out
of the Java heap. Your application might be distributed across multiple boxes
that are connected by a network, rather than a distributed shared memory. It is
possible that your your data structures do not fit into available physical
memory, and so, rather than relying on the operating system's paging
functionality, you may find that you must implement a facility for doing so more
efficiently. Sometimes, you have enough physical memory, but are constrained by
available address space.\index{Address space exhasution} This is especially
likely to happen if you are running on a 32-bit \jre; e.g. if you run with a 1.5
gigabyte Java heap, and have one hundred thousand more more loaded classes, you
run the risk of exhausting the 2 gigabyte limit that most 32-bit operating
systems have, for each process. On many 32-bit versions of the Windows operating
system, and on 31-bit IBM mainframes, the address space limit is even lower than
2 gigabytes. It might also be that you have a need to perform relational queries
against your Java data structures, and are considering storing them in a
relational store to avoid having to implement the querying logic yourself. Your
reality might also be some combination of these.

It is important to remember that, in Java, the terms marshalling and
serialization are not synonymous. Serialization is the task of writing out the
bits, and the form of those bits. Marshalling, in the Java lexicon, is
serialization plus the assurance that all parties in the communication can agree
upon the version of the code that generated those bits. Without this agreement,
it can become difficult to manage evolving code bases, in the face of legacy
installations. This disparity of terminology is somewhat unique to Java, but the
distinction is nonetheless important. 

This book does not address the issues of compatibility of code bases, other than
to observe that, if your use case does not require handling code skew, then you
have options that will perform much better than using the built-in Java
marshalling mechanisms. In Java, there are several commonly available mechanisms
for marshalling your objects into and out of the Java heap. These include the
built-in Java object serialization, the Apache \class{XMLSerializer} and
\class{XMLDeserializer}. There is also a variety of libraries that provide a
mapping between Java objects and a relational backing store; an example is
RedHat's \class{Hibernate}. All of these come with a fair amount of expense,
because the \emph{operational} form of the data, that is the way the bits are
laid out as your Java code operates on them, is different from the serialized
form. Therefore, use of these serialization libraries usually entails an
expensive translation between two disparate storage formats.



\section{Memory mapping}
\label{sec:memory-mapping}
\index{Memory Mapping}

One way to bring data in and out of Java without paying a marshalling expense
is via memory mapped I/O. When you \emph{memory map} a file into your address
space, you can treat the file as if it were an array. Reads and writes to the
array are reflected as disk reads or writes, and these operations are usually
done at a page granularity. Actual disk I/O may not occur with every array
access. This is the case if the operating system decides that it has enough
physical memory to keep the written pages in memory, and you haven't specified
that array writes should be written through to disk every time. Reads may be
serviced from this cache, as well. In this way, memory mapped I/O can have the
benefit of well-tested caching that balances that performance needs of all
processes running on your machine, without any work on your part.  Memory mapping
is a common trick used that is used by C programmers seeking the ultimate in
performance.

Since the cache is managed by the operating system, cached pages persist across
process boundaries. Therefore, if your application runs as multiple steps, each
in a separate process, then storing your data structures in memory mapped files
can result in a combination of caching and serialization-free persistence. The
unwritten buffers may eventually find their way to disk (if the underlying file
is not deleted first), but this needn't happen when one process terminates.

Used to its utmost, memory mapping can additionally offer one-copy bulk
transfers of memory to and from other processes, disk, or the network. For
example, say your application takes data from the network and writes it to
disk. If you memory map the network input buffers and the output file, and issue
bulk transfers from the input to the output, then it is possible that the
operating system will transfer the data directly from the network buffers to the
disk buffers, without first copying them out of the kernel, or into the
Java heap.

\paragraph{Memory Mapping in Java}
\index{\code{java.nio} library}
As of Java 1.4, a memory mapping facility is provided by the \code{java.nio}
library. This Java library manifests data as \class{ByteBuffer} objects. This
interface mimics an array, providing random access and bulk \code{get} and
\code{put} methods, but no insertion or deletion operations. Using this common
interface, you can access four repositories of data: network transmissions, files
on disk, memory allocations in the native heap, and allocations in the Java heap.
While the latter two can serve only as transient repositories for your data, they
avoid the expense of having to create a file on disk --- an expensive operation,
on most file systems, if done frequently.

You may find it useful to have the option to have some data stored in transient
storage, and others in persistent storage, backed by files on disk, and interact
with both using the same API. The \code{java.nio} library lets you do this. An
important advantage of using native \class{ByteBuffer} storage, over Java heap
storage, is that your application can run on arbitrarily large inputs without the
constraints of a fixed-maximum size Java heap.

No matter where the data is stored, you also have a choice of whether to use
standard Java byte ordering, or the byte ordering of the platform on which the
application is executing. Of course, this only matters if the data values you are
accessing are larger than a byte. On top of a \class{ByteBuffer}, you can layer
other primitive-type views. For example, the instance method
\code{ByteBuffer.asIntBuffer()} returns an \class{IntBuffer} that takes care of
any bit manipulations that are necessary to access the data as Java integers. If
you have chosen to use native byte ordering, then accessing the elements of an
\class{IntBuffer} on a 32-bit \jre requires no bit manipulation.

An example use of this library to create an integer array-like view over a file
is:

\begin{shortlisting}
IntBuffer mapAsIntegers(File file) {
   ByteBuffer buffer = new RandomAccessFile(file).getChannel().map(MapMode.READ_WRITE,0,file.length());
   // use platform, not Java, byte ordering
   buffer.order(ByteOrder.nativeOrder());
   return buffer.asIntBuffer();
}
\end{shortlisting}

\paragraph{Memory Mapping Your Column-oriented Storage}

There is a beneficial combination of memory mapping and a column-oriented
approach to storing your data. The interface to all memory mappings is an
array, and a column-oriented storage structure stores data as arrays. The
update, from the previous \class{EdgeModel} is easy, but requires a notion of a
namespace for each edge model. The namespace is necessary so that the model can
be mapped in from disk:

\begin{shortlisting}
class EdgeModel {
   IntBuffer from, to;
   public EdgeModel(File namespace) { // constructor
      from = mapAsIntegers(new File(namespace, "from"));
      to = mapAsIntegers(new File(namespace, "to"));
   }
   public int getEdgeFrom(int i) {
      return from.get(i);
   }
   public int getEdgeTo(int i) {
      return to.get(i);
   }
}
\end{shortlisting}

\paragraph{Difficulties with Memory Mapping in Java}

Each memory mapped file consumes only as much \emph{physical} memory as is
available and which the operating system decides, in its balancing act, is worthy
to allocate to the mapping. While all major operating system manage consumption
of physical memory, they do not similarly manage address space consumption. Thus,
each mapped \class{ByteBuffer} consumes a swath of address space equal to the
\emph{full} size of the map. For this reason, if you decide to use memory
mapping, you will run into several pernicious and interconnected problems. These
problems are orthogonal to any problems that stem from a choice to use
column-oriented storage.

The primary problem comes from the lack of an unmapping facility in the
\code{java.nio} library. You can not explicitly unmap a mapped area. Instead,
and, quite oddly, in direct contradiction of widely published best practices, the
library takes care of unmapping in the \code{finalize} method of the mapped
\class{ByteBuffer}. This leaves the timing of unmapping at the whim of garbage
collection and the subsequent scheduling of the finalizer thread. If you
application allocates very little in the way of temporary objects, but uses
memory mapping extensively, you will have failures due to address space
exhaustion. The garbage collector won't run, because plenty of Java heap is
available for allocation. However, memory mappings fail, because the process's
address space is fully consumed by older mappings, many of which would be
unmapped if the garbage collector were only to run.

This implementation decision has a strong negative interaction with the second
problem: the \jre does not, upon address space exhaustion, attempt to run a
garbage collection and finalization pass in order to clear out mapped byte
buffers. Therefore, you may suffer from \class{OutOfMemoryException} failures,
due to running out of address space, despite the fact that many of your mapped
regions are actually ready to be unmapped.

The third problem you may encounter is primarily to be found on Windows
platforms. The Windows operating system does not allow a file to be removed from
the file system if there are existing memory maps over any part of the file.
This, combined with the inability to explicitly unmap a mapping, can lead to a
situation where files remain on disk when you no longer need them. For example,
this can happen if there is a point in your code where you know the file is no
longer necessary, but there exist references from live objects or the stack to
the \class{ByteBuffer} object that represents the memory mapping. Since the
\class{ByteBuffer} facade is not garbage collectible, then its \code{finalize}
method will not be called, and hence the unmapping will not occur.

There is a set of policies you can follow to reduce the likelihood of such
problems. First, if possible, you should implement a correlated lifetime pattern
for the \class{ByteBuffer} objects. If these objects become garbage collectible
as soon as a phase or request completes, or correlated object is collected, then
the \code{ByteBuffer.finalize} method will be called as soon as possible after
that correlated event occurs. Next, you must trap all memory mapping failures,
force a garbage collection and finalization pass, and then retry the memory
mapping; doing this several times in a loop is recommended. Third, on some
versions of the \jre, you will find that a memory mapping failure results in the
\jre terminating the process. This was one, by the \jre developers, with the
thought that a memory mapping failure was a sign of catastrophic failure. You
know now that this is not the case, it is rather a symptom of the overall poor
design of the \code{java.nio} library. To work around this problem, you can set
up a security policy that disables calls to \code{System.exit}, though you must
be careful that your own code does not rely on this call.

The last good design principal, when using memory mapping in Java, is to rely on
file-based memory mappings as little as possible. If you need only temporary
mappings, then consider using the anonymous mappings described earlier:
\begin{shortlisting}
IntBuffer allocateAnonymousInts(int numBytes) {
   ByteBuffer buffer = ByteBuffer.allocateDirect(numBytes);
   buffer.order(ByteOrder.nativeOrder());
   return buffer.asIntBuffer();
}
\end{shortlisting}
When using anonymous maps, you must be aware of the default limits on these
native allocations. With Sun \jres, you can configure the maximum allowed number
of bytes allocated in this way via the command line argument
\code{-XX:MaxDirectMemorySize}; this option takes the same arguments as the
\code{-Xmx} setting. With IBM \jres, you do not need to specify a maximum value.


