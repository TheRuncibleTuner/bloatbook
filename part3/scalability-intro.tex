\chapter{Assessing Scalability}
\label{chapter:scalability-intro}

Despite heroic efforts at tuning classes and optimizing the use of collections,
you may still be unable to fit your application into available physical memory
or address space. Sometimes, heroic efforts are not even possible, because this
problem was discovered very late in the development cycle. If you don't discover
till the final stages of testing that your application won't fit, you will have
difficulty finding the resources to perform extensive tuning. Optimizing earlier
in the development process would help, but these things need to be budgeted.
Resources spent on earlier tuning are resources taken away from other aspects of
development, such as architectural design and coding. To avoid wasting time
blindly tuning everything, it is helpful to have a quick way of knowing whether
an inefficient data structure will really matter in the grand scheme of things.
A particular structure might have a bloat factor of 95\%, being composed of only
5\% actual data, but if that structure only contributes to a few percent of
overall memory consumption, who cares?

You can focus your tuning efforts by adopting a design strategy that assesses
the \emph{scalability} of your data structures. By focusing on scalability, you
can answer two kinds of questions:

\begin{itemize}
  \item \textbf{Will It Fit?} As you scale up your application, adding more
  users for example, it'd be nice to know whether it will fit
  within a given memory budget. 
  %This requires knowing the \emph{variable cost} of each user.
  \item \textbf{Should I Bother Tuning?} Your tuning efforts should focus 
on those structures that can benefit from the tuning tasks described in
the \autoref{part:1}. 
%This requires knowing how how bloat factor changes as a data structure grows.
\end{itemize}

The rest of this part of the book will help you in answering these
questions, and in developing solutions for the cases when the answers to both
questions are no. If so, then you need to consider stepping outside of the Java
box.

\paragraph{Will My Design Fit?}
\autoref{chap:estimating-scalability} introduces a metric that you can use to
estimate the answers to these questions. The metric, called the  Maximum Room
For Improvement, quantifies how much your design can benefit from tuning. From
this number, you can extrapolate how much memory will be required to fit your
data. If the room for improvement is high, then you should consider
applying the tuning regimen detailed in \autoref{part:1}.

If your design doesn't fit in physical memory constraints, but you still have
address space available (see \autoref{sec:address-space}), then the first
solution you should consider is buying more physical memory. If your budget
allows for this, then by all means you should strongly consider doing so.
% There are some downsides to keep in mind. The first is heap size limits. At
% some point, you will need to switch from a 32-bit JVM to a 64-bit
% JVM\index{64-bit}. This switch will result in a large increase in overhead. As
% discussed earlier, the amount of blowup resulting from a switch to a 64-bit
% JVM depends on the degree of delegation in your data models. A 64-bit JVM only
% adds overhead to the extent that your data models have headers and pointers.
% The alternative, running with compressed references\index{Compressed
% References}, limits you to around 32GB of memory, and imposes a few percent
% slowdown. In addition, running with a large heap can result in increases in
% the length and fluctuations in pause times.


\paragraph{Making it Fit by Breaking the Java Mold} Despite being an
object-oriented language, there are still some tricks you can use that let you
continue to program in Java, but store objects in a non-object oriented way.
These tricks come at the expense of some xtra programming time and maintenance
expense, but can dramatically increase the scalability of your data models.
\autoref{chapter:large-long-lived} describes these \emph{bulk storage}
techniques.
%It is possible to store data only in arrays, as one would in
%a language such as Fortran, and retain a great deal of object orientation in
%your data models and programming interfaces.

\paragraph{It Still Won't Fit}
% Shuttle Objects in and Out of the Java Heap}
Rather than attempting to fit all of your objects into a single heap, you can
store them outside of the Java heap, and swap them in (and out), on demand. If
the logic of your application allows for recomputing the data stored in these
objects, you need to be careful to compare the recreation cost with the costs of
marshalling objects. You can choose to marshall objects to and from a local
disk, or you can use one of several frameworks that provide a distributed
key-value map.
